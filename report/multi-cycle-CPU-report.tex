\documentclass{article}
\usepackage[UTF8]{ctex}
\usepackage{geometry}
\usepackage{listings}
\usepackage{fancyhdr}
%\usepackage[colorlinks,linkcolor=blue]{hyperref} 

%\usepackage{amsmath}
\usepackage{graphicx}
%\usepackage{pgfplots}

\geometry{left = 2cm,right = 2cm,top = 3cm,bottom = 3cm}
\pagestyle{fancy}
\fancyhf{}
\fancyhead[R]{\leftmark}
\fancyhead[L]{\rightmark}
\fancyhead[C]{多周期处理器}
\fancyfoot[C]{\thepage}
\renewcommand{\headrulewidth}{0.2mm}

\title{数字逻辑与处理器基础\\多周期处理器}

\author{cvxbzn}

\begin{document}

\maketitle

\section{数字通路设计}
\section{寄存器与多路选择器及其功能}

\section{控制信号分析与有限状态机实现}
\subsection{控制信号及具体功能}
\subsection{状态转移图}

\section{ALU功能拓展}
\subsection{setsub类型和机器码字段内容}
\subsection{ALU verilog代码修改}
\subsection{仿真结果}

\section{汇编程序分析-1}
\subsection{计算寄存器值}
\subsection{仿真结果}

\section{汇编程序分析-2}
\subsection{程序功能以及代码注释}
\subsection{将这段汇编翻译成机器码并写出}
\subsection{\$a0,\$v0值}
\subsection{观察、描述并解释 PC、\$a0、\$v0、\$sp、\$ra 如何变化}

\section{异常处理}

%\begin{lstlisting}[language=Matlab]
%\end{lstlisting}

\end{document}